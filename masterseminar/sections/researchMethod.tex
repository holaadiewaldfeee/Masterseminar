\section{Research Methodology}\label{research}

First, this paper deals with research on the keywords "fault-proneness", "fault-proneness in object-oriented systems" and "fault prediction metrics" in google scholar. To get more literature on this topic, the keywords linked in the papers were used as further search. The linked book "Empirical Software Engineering" by Springer Verlag appeared frequently with articles like for example "\textit{Fault prediction modeling for software quality estimation: Comparing commonly used techniques}" \cite{khoshgoftaar2003fault}. 

Among them keywords there are "object-orientation", "object-oriented design metrics" and "faulty classes" to get the content background of object orientation. As first an overview was created in such a way, in order to be able to classify the term "fault-proneness" into the software development process. The top listed papers that were suggested in google scholar were either sorted out or read more closely by reading the abstract. If the abstract sounded interesting for the topic, then the introduction was read and the conclusion. In some papers, such as the research on metrics, the exact procedure and the analysis section were also read in detail. Two of the metrics were then chosen as a showcase model to illustrate fault-proneness in software systems and how to avoid faults. Further literature was researched for the content background to summarize the basic knowledge about object orientation and a simple understanding of objects, classes and methods. 

At the end of research, some conclusions of papers of metrics were compared to summarize the equalities and differences, as well as the benefits and limitations of the object-oriented design metrics just considered.



