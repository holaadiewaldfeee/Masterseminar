\section{Conclusion}\label{conclusion}

In the future, as well as today, fault-proneness estimation and prediction could play a key role in software product quality control.  In this area, much effort has been invested in defining metrics and identifying models for system evaluation, object-oriented models as well as others.  Using these metrics to assess which parts of the system are more fault-proneness is of primary importance, as summarized in this paper.
Much work has concentrated on how to select the software metrics that are most likely to indicate fault-proneness. 

In the field of software evolution, metrics can be used for identifying stable or unstable parts of software systems.

Moreover the question is, what can really prevent faults in the later used system beforehand?

If you take up the other software trends of development mentioned at the beginning, it becomes clear that they are all connected in a certain way. The fault-proneness and the maintenance grow with increasing complexity.Ensuring security and quality always involves a number of features.

%- summarize discussion and results; name important metrics/properties that should stay in mind after reading this

%- what can reduce the fault-pron in software development

%- outlook future work (future work wäre größere systeme anschauen und testen auf faulty classes und metriken testen an sich)

%- This paper discussed the concept of fault proneness using the example of object-oriented programming and design metrics, which means that for the most part the results can only be transferred to the object-oriented context.


%- final catch and the link to the superordinate context (research trends in software technology); link to introduction and motivation

%wie in related work aufgezeigt die flexible metrics die können als ausblick hilfreich sien wenn sich der kontexxt leicht ändert