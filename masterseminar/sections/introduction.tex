\section{Introduction}

In the field of software development there are some keywords like security, consistency or reliability that are indispensable today. Everyone wants the best and most intelligent software, but with increasing complexity the possibility of fault-proneness in the software increases. In the following, the aforementioned topic is examined under the programming language model of object orientation and metrics that can be considered, which is an important topic in research trends in software technology nowadays.
One might think that testing the software and the resulting errors is one of the last steps in the software development process, but this is a false assumption. The earlier the system is examined and tested for critical points, the more work will be saved in later, more cost-intensive development steps.
The conditional probability that a class in software systems contains a fault, given the metrics for that class, is called fault-proneness.
In the object oriented context there are a few constructs like class, coupling, cohesion, inheritance, information hiding and polymorphism, which influence the fault-proneness. 
One of the most common aspects incorporated into metrics is that of coupling, which refers to the degree of direct knowledge that one element has of another. The degree of interconnectedness of the whole system is a key element in the software development, for example it is important how big the effect of changing one attribute of one class has on all others and especially how many.
The coup´ling of the classes in the system plays a central role in the effects of software faults.
Several studies (hier zitieren) have been carried out to determine which metrics are useful in capturing important quality attributes such as fault-proneness  and fault prediction, which are summarized in the following sections.

- the main research objective is the consolidation of some metrics and methods related to fault-proneness in the software development process


This paper is organized as follows: Section \ref{research} presents research methodology and background as well as related work. In Section \ref{content} the content background is described. Section \ref{analysis} contains the analysis of design metrics and the results of some empirical studies. After the investigation of the effects on the fault-proneness metrics, a discussion follows, including the classification of the relevance in the software development process and the parties involved. The paper is concluded and gives a future outlook in Section \ref{conclusion}.