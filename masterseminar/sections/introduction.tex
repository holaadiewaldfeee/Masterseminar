\section{Introduction}

In the field of software development there are some keywords like security, consistency or reliability that are indispensable today. Everyone wants the best and most intelligent software, but with increasing complexity the possibility of fault-proneness in the software increases. In the following, the aforementioned topic is examined under the programming language model of object orientation and metrics that can be considered, which is an important topic in research trends in software technology nowadays.
One might think that testing the software and the resulting errors is one of the last steps in the software development process, but this is a false assumption. The earlier the system is examined and tested for critical points, the more work will be saved later, more cost-intensive development steps.

A software fault is defined as an anomalous condition or defect at the component, device, or subsystem level that can lead to a failure. Therefore an undesirable companion of developing, where the objective is to appear as little as possible.
Fault-proneness is an important external software quality attribute of interest to software developers and practitioners. The fault-proneness of an object-oriented class indicates the extent to which the class, given the metrics for that class, is fault-prone. Since it is difficult to measure the fault-proneness of software that is not yet in use, predictive models are applied to estimate the fault-proneness of software classes.

Several studies like "\textit{Fault-Proneness of Open Source Software: Exploring its Relations to Internal Software Quality and Maintenance Process}" \cite{kozlov2013fault} or "\textit{Empirical analysis for investigating the effect of object-oriented metrics on fault proneness: a replicated case study}" \cite{b2aggarwal2009empirical} have been carried out to determine which metrics are useful in capturing important quality attributes such as fault-proneness  and fault prediction, which are summarized in the following sections.
Furthermore the main research objective is the consolidation of some metrics and methods related to fault-proneness in the software development process.
% Ich würde hier nicht die Titel auflisten sondern er schreiben Author A and others researched to X,Y in there Paper. They came to the following results ... [1].
% Titel nicht schreiben, et a.. blabla Autoren et al. [1]..

This paper is organized as follows: Section \ref{content} deals with the content background of design metrics in object-oriented systems and the basic technologies about object orientation. In Section \ref{research} the research methodology is described. Afterwards Section \ref{related} presents some studies and further literature that handle the concept of fault prediction and further analysis in that topic area. After the related work, it comes to Section \ref{analysis}, that contains the analysis and summary of obejct-oriented design metrics and the results of some empirical studies that are concerned with fault-proneness and that used to support the software development process. After the investigation of the effects on these metrics, a discussion follows in Section \ref{discussion}, including the classification of the relevance in the software development process and the parties involved. Furthermore there are limitations and benefits of the considered metrics. Finally, the paper is concluded and gives an outlook in Section \ref{conclusion}.