\documentclass[conference]{IEEEtran}
\IEEEoverridecommandlockouts

\usepackage{cite}
\usepackage{amsmath,amssymb,amsfonts}
\usepackage{algorithmic}
\usepackage{graphicx}
\usepackage{textcomp}
\usepackage{xcolor}
\usepackage{tabularx}
\def\BibTeX{{\rm B\kern-.05em{\sc i\kern-.025em b}\kern-.08em
    T\kern-.1667em\lower.7ex\hbox{E}\kern-.125emX}}
\begin{document}

\title{Metrics About Fault-Proneness In Object-Oriented Systems}

\author{\IEEEauthorblockN{Sabrina Böhm}
\IEEEauthorblockA{\textit{Universität Ulm} \\
Ulm, Germany \\
sabrina.boehm@uni-ulm.de}}

% Notizen:
% - Vortrag auf deutsch ~20min
% - 12 Seiten Inhalt ohne Refs
% - zwischen 8 und 25 Refs 
% 
% TODO - im abstract teasern auf welche metriken man genau eingeht
% TODO - mehr im Intro aus abstract wiederholen, mehr sagen warum Fehlerauswirkungen kritisch sind, Auswirkungen von fehlern wie im abstract tiefer
% TODO - im Intro auch schon zitieren wenn man muss, study such as [1], wenn man genaue Studien anspricht
% TODO - im Intro was mach ich jetzt genau im folgenden noch mehr sagen
% TODO - Research objective ganz am Ende ins Intro verschieben
% TODO - und research objective in research methodolgy wieder aufgreifen (wie hab ich die Metriken gefunden, wie bin ich vorgegangen um sie zu vergleichen)
% TODO - argumentieren warum ich die 2 metriken rausgepickt hab

\maketitle

\begin{abstract}
	Software quality is becoming increasingly important in modern times. Faulty or insufficient software can have severe consequences. For this reason, aspects such as qualitiy, security, reliability and maintainability must be considered at an early stage of the development process. Early detection of bugs or problems in the software prevents enormously high costs at the later times. In safety-critical areas, for example, the property of fault-proneness must be carefully considered. In order to include this aspect early in the development process, there are a number of metrics and methods that can estimate the fault prediction or reduce the fault-proneness of the software. By following certain rules and procedures, high costs and time can be saved for both developers and consumers.
	The goal of this work is to illustrate some of the widely used methods and design metrics that consider fault-proneness in object-oriented systems.
	
\end{abstract}

\section{Introduction}

In the field of software development there are some keywords like security, consistency or reliability that are indispensable today. Everyone wants the best and most intelligent software, but with increasing complexity the possibility of fault-proneness in the software increases. In the following, the aforementioned topic is examined under the programming language model of object orientation and metrics that can be considered, which is an important topic in research trends in software technology nowadays.
One might think that testing the software and the resulting errors is one of the last steps in the software development process, but this is a false assumption. The earlier the system is examined and tested for critical points, the more work will be saved in later, more cost-intensive development steps.

A software fault is defined as an anomalous condition or defect at the component, device, or subsystem level that can lead to a failure. Therefore an undesirable companion of developing, where the objective is to appear as little as possible.
Fault-proneness is an important external software quality attribute of interest to software developers and practitioners. The fault-proneness of an object-oriented class indicates the extent to which the class, given the metrics for that class, is fault-prone. Since it is difficult to measure the fault-proneness of software that is not yet in use, predictive models are applied to estimate the fault-proneness of software classes.

Several studies like "\textit{Fault-Proneness of Open Source Software: Exploring its Relations to Internal Software Quality and Maintenance Process}" \cite{kozlov2013fault} or "\textit{Empirical analysis for investigating the effect of object-oriented metrics on fault proneness: a replicated case study}" \cite{b2aggarwal2009empirical} have been carried out to determine which metrics are useful in capturing important quality attributes such as fault-proneness  and fault prediction, which are summarized in the following sections.
Furthermore the main research objective is the consolidation of some metrics and methods related to fault-proneness in the software development process.

This paper is organized as follows: Section \ref{content} deals with the content background of design metrics in object-oriented systems and the basic technologies about object orientation. In Section \ref{research} the research methodology is described. Afterwards the Section \ref{related} presents some studies and further literature that handle the concept of fault prediction and further analysis in that topic area. After the related work, it comes to the Section \ref{analysis}, that contains the analysis and summary of obejct-oriented design metrics and the results of some empirical studies that are concerned with fault-proneness and that used to support the software development process. After the investigation of the effects on these metrics, a discussion follows in Section \ref{discussion}, including the classification of the relevance in the software development process and the parties involved.Furthermore there are limitations and benefits of the considered metrics. Finally, the paper is concluded and gives an outlook in Section \ref{conclusion}.
\section{Content Background}\label{content}

In the following, we will consider fault-proneness as already mentioned, but from a restricted point of view, in an object-oriented context. Many software systems in use are based on object-oriented design. This means that data and program code are encapsulated in reusable objects. Everything is based on the communication of objects. For this purpose, classes, interfaces and methods, as well as attributes are declared and thus serve to represent states. This structure alone protects against fault-proneness, since the code is reusable and thus the programming effort is reduced. Thus, fewer errors occur. 

But why should a programmer actually pay attention to fault-proneness at all? For different programming paradigms and programming constructs different rules apply which must be considered in relation to fault-proneness.  The restriction on object orientation is to facilitate the understanding. In addition many principles are contained, which are taken up in other programming concepts again. Thus some conclusions which are drawn here are also differently realizable and applicable to other software concepts.

When software errors are made, it is often not only tedious to find the programming error, but also expensive. When software errors are made, it is often not only tedious to find the programming error, but also expensive. In large systems that are used by many people every day, a small error can cost millions. When software errors are made, it is often not only tedious to find the programming error, but also expensive. In large systems that are used by many people every day, a small error can cost millions. A common misconception is that fault-proneness should only be considered at the end of the software development process. Especially at the beginning of the development you should build the software architecture in a way that it is less error-prone. Changes in the architecture are always more expensive later. In addition, even before the programming itself begins, attention should be paid in the planning to various concepts and metrics, which are shown below.


- fault content, fault proneness, fault prediction background

- relation to other aspects like reliability, correctness, completeness, maintainability etc.

- to what extent can you do analysis now and what should you consider when developing object-oriented

- from the large overall background to the more detailed topic of fault proneness $\rightarrow$ transition to metrics
\section{Research Methodology}\label{research}

- wie bin ich vorgegangen um literatur zu finden

- google scholar, was waren meine Kriterien und was waren meine Abbruchbedingungen, hab ich nur das abstract gelesen oder das ganze Intro oder erste 5 Seiten oder 100 gelesen etc., wie hab ich die paper "abgegrast"

- Wo hast du nach den Terms gesucht? Titel? Abstract?

- research method/background/strategy and related work, research objectives 

- the way I came across literature: google scholar and there recursively linked further literature and papers on terms like "fault-proneness", "fault-proneness in object-oriented systems", "fault prediction metrics", so on

- discuss findings in a chronological manner

- \cite{b13singh2010empirical} Empirical validation of object-oriented metrics for predicting fault proneness models

- research background: summary of interpretation of previous research and what this paper is about depending on the research
\section{Related Work}\label{related}

To take a close look at fault-proneness in object oriented systems, basic oo knowledge is important, as explained in section \ref{content}. The paper "Beyond Language Independent Object-Oriented Metrics: Model Independent Metrics" deals not only with the concept of object orientation but also beyong languages \cite{lanza2002beyond}. The aspects of object orientation such as class, method and attribute can be found here as shown in the following tables, which are subdivided into individual metrics. The meta model from figure \ref{fig0} serves here as a template to understand the relationships between the individual metrics. In the following tables \ref{tab:classmetrics},\ref{tab:methodmetrics} and \ref{tab:attributesmetrics} the abbrevations and the associated metrics are listed. The used meta model and the metrics of the tables allow a multiple extension into different research directions. Thus, they fit the principle of object orientation and serve as a basic template to get into the basic structure of metrics. The advantages of this approach are the increased flexibility, i.e.,
new metamodels can be introduced from any context (for example, the financial world or databases), which provides a standard metric without the need to implement new metrics every time a new context is introduced \cite{lanza2002beyond}. Now, if you look at this system completely from an object orientation perspective, you can see that the basic programming concepts are translated into metrics. Each new method or variable crates more space to generate faults. The complexity of a class is depends, i.e., on the number of methods (NOM) or the number of attributes (NOA), that can be calculated with $(NOA = NIV + NCV)$, as shown in table \ref{tab:classmetrics}. An important metric about the methods of a class is the number of input parameters (NOP) or the number of access on attributes (NMAA), as you can see in figure \ref{tab:methodmetrics}. For the attribute metrics, the number of direct hits is of great importance, as can be found in figure \ref{tab:attributesmetrics}. 

\begin{table}
	\caption{Class metrics from the meta model.}~\label{tab:classmetrics}
	
	\setlength\tabcolsep{3pt}
	\renewcommand{\arraystretch}{1.4}% for the vertical padding
	\begin{tabularx}{\columnwidth}{ | c | p{7cm} | }
		\hline
		Abbrevation & Description \\ \hline\hline
		HNL & Number of classes in superclass chain of class \\ \hline
		NAM & Number of abstract methods \\ \hline
		NCV & Number of class variables \\ \hline
		NIA & Number of inherited attributes \\ \hline
		NIV & Number of instance variables \\ \hline
		NME & Number of methods extended, i.e., redefined in subclass by invoking the same method on a superclass \\ \hline	
		NMI & Number of methods inherited, i.e., defined in superclass and inherited unmodified by subclass\\ \hline
		NMO & Number of methods overridden, i.e., redefined compared to superclass\\ \hline
		NOA & Number of attributes $(NOA = NIV + NCV)$ \\ \hline
		NOC & Number of immediate subclasses of a class \\ \hline
		NOM & Number of methods\\ \hline
		PriA & Number of private attributes (equivalent for protected and public attributes)\\ \hline
		PriM & Number of private methods (equivalent for protected and public attributes)\\ \hline
		WLOC & Sum of all lines of codes over all methods \\ \hline
		WMSG & Sum of message sends in a class\\ \hline
		WNMAA & Number of all accesses on attributes\\ \hline
		WNOC & Number of all descendant classes\\ \hline
		WNOS & Sum of statements in all method bodies of class\\ \hline
		WNI & Number of invocations of all methods \\ \hline
	\end{tabularx}
\end{table}

\begin{table}
	\caption{Method metrics from the meta model.}~\label{tab:methodmetrics}
	
	\setlength\tabcolsep{3pt}
	\renewcommand{\arraystretch}{1.4}% for the vertical padding
	\begin{tabularx}{\columnwidth}{ | c | p{7cm} | }
		\hline
		Abbrevation & Description \\ \hline\hline
		LOC & Method lines of code \\ \hline
		NMA & Number of methods added, i.e., defined in subclass and not in superclass \\ \hline
		MSG & Number of method messages send \\ \hline
		NOP & Number of input parameters \\ \hline
		NI & Number of invocations of other methods within method body \\ \hline
		NMAA & Number of access on attributes \\ \hline
		NOS & Number of statements in method body \\ \hline
	\end{tabularx}
\end{table}

\begin{table}
	\caption{Attribute metrics from the meta model.}~\label{tab:attributesmetrics}
	
	\setlength\tabcolsep{3pt}
	\renewcommand{\arraystretch}{1.4}% for the vertical padding
	\begin{tabularx}{\columnwidth}{ | c | p{7cm} | }
		\hline
		Abbrevation & Description \\ \hline\hline
		AHNL & Class HNL in which attribute is defined \\ \hline
		NAA & Number of times directly accessed \\ \hline
	\end{tabularx}
\end{table}

The limitations of this approach are that not all object-oriented software metrics can be defined in terms of the language independent meta model. Certain metrics tend to be very specialized and are therefore difficult to define in a generic way. Another limitation of this basic concept is that for some metrics, there is it is not yet known how best to define them in a generic way, so the meta model does not include coupling metrics and cohesion metrics.

Fault-Proneness of Open Source Software: Exploring its Relations to
Internal Software Quality and Maintenance Process \cite{kozlov2013fault}. The strengths of our study include the following: 1) we stud-
ied a greater number of metrics than most of the related studies, 2) we studied a greater number of OSS-systems than most
of the studies, and 3) we focused on the fault-proneness of modern Java-based systems and investigated them as an aggre-
gated sample.
ibutes contributing to fault-proneness of open source
software (OSS) can be explained in terms of internal quality
attributes and maintenance process metrics (maintenance
being a phase of the OSS life cycle).
This has been an empirical multiple case study. We have
explored to what extent and how fault-proneness could be
explained by means of internal quality attributes and mainte-
nance process metrics. We first conducted a literature survey
as a basis for taking into account the main findings of other
researchers. Next we analyzed eight OSS systems and their
342 releases.
Software quality was measured in terms of 76 internal
quality attributes, using the static analysis tool SoftCalc.
There were 23 maintenance process metrics obtained from
Source Forge Issue Tracking System (SFITS). Fault-
proneness was measured in terms of Rate of Bug Reports
(RBR) and Average Bug Priority Level (ABPL). That data
was also obtained from SFITS.

The Impact of Accounting for Special Methods in the Measurement of
Object-Oriented Class Cohesion on Refactoring and Fault Prediction
Activities ::: \cite{b4al2012impact}
Class cohesion is a key attribute that is used to assess the design quality of a class, and it
refers to the extent to which the attributes and methods of the class are related. Typically,
classes contain special types of methods, such as constructors, destructors, and access
methods. Each of these special methods has its own characteristics, which can artificially
affect the class cohesion measurement.
This paper empirically explores the impact of including or
excluding special methods on cohesion measurements that were performed using 20
existing class cohesion metrics.
study: The empirical study applies the metrics that were
considered to five open-source systems under four different scenarios, including (1)
considering all special methods, (2) ignoring only constructors, (3) ignoring only access
methods, and (4) ignoring all special methods.
results: he results of
the empirical studies show that the cohesion values for most of the metrics considered
differ significantly across the four scenarios and that this difference significantly affects
the refactoring decisions, but does not significantly affect the abilities of the metrics to predict faulty classes.
This paper empirically addressed whether to include or exclude special methods from
cohesion measurements. Two types of special methods were considered, constructors and
access methods.


Software fault prediction metrics: A systematic literature review ::: \cite{b7radjenovic2013software}. Context: Software metrics may be used in fault prediction models to improve software quality by predict-
ing fault location.
Objective: This paper aims to identify software metrics and to assess their applicability in software fault
prediction. We investigated the influence of context on metrics’ selection and performance.
Method: This systematic literature review includes 106 papers published between 1991 and 2011. The
selected papers are classified according to metrics and context properties.
Results: Object-oriented metrics (49%) were used nearly twice as often compared to traditional source
code metrics (27%) or process metrics (24%). Chidamber and Kemerer’s (CK) object-oriented metrics were
most frequently used. According to the selected studies there are significant differences between the
metrics used in fault prediction performance. Object-oriented and process metrics have been reported
to be more successful in finding faults compared to traditional size and complexity metrics. Process met-
rics seem to be better at predicting post-release faults compared to any static code metrics.
Conclusion: More studies should be performed on large industrial software systems to find metrics more
relevant for the industry and to answer the question as to which metrics should be used in a given
context.In this paper we performed a validation of object-oriented design metrics on a commercial Java system.
The objective of the validation was to determine which of these metrics were associated with fault-
proneness, and hence can be used for predicting the classes that will be fault-prone and for estimating
the overall quality of future systems. Our results indicate that an inheritance and an export coupling
metric were strongly associated with fault-proneness. Furthermore, the prediction model that we
constructed with these two metrics has good accuracy, and the method we employed for predicting the
quality of a future system using design metrics also has a good accuracy.
While this is a single study, it does suggest that perhaps there are a small number of metrics that are
strongly associated with fault-proneness, and that good prediction accuracy and quality estimation
accuracy can be attained. This conclusion is encouraging from a practical standpoint, and hence urges
further studies to corroborate (or otherwise) our findings and conclusions.



A Validation of Object-Oriented Design Metrics as Quality Indicators ::: his paper presents the results of a study in which we empirically investigated the suite of object-oriented design metrics introduced in. More specifically, our goal is to assess these metrics as predictors of fault-prone classes and, therefore, determine whether they can be used as early quality indicators. This study is complementary to the work described in [30] where the same suite of metrics had been used to assess frequencies of maintenance changes to classes \cite{b11basili1996validation}.

The prediction of faulty classes using object-oriented design metrics ::: \cite{b10el2001prediction} Contemporary evidence suggests that most field faults in software applications are found in a small percentage of the software's components. This means that if these faulty software components can be detected early in the development project's life cycle, mitigating actions can be taken, such as a redesign. For object-oriented applications, prediction models using design metrics can be used to identify faulty classes early on. In this paper we report on a study that used object-oriented design metrics to construct such prediction models. The study used data collected from one version of a commercial Java application for constructing a prediction model. The model was then validated on a subsequent release of the same application. Our results indicate that the prediction model has a high accuracy. Furthermore, we found that an export coupling (EC) metric had the strongest association with fault-proneness, indicating a structural feature that may be symptomatic of a class with a high probability of latent faults.


Assessing the Applicability of Fault-Proneness Models Across Object-Oriented Software Projects ::: Furthermore a number of papers have investigated the relationships between metrics of design and the metrics that detect faults in object oriented software. One of the main objectives of this paper is to assess whether fault-proneness models, based on design measurement, are applicable and can be viable decision making tools when applied from one object-oriented system to the other, in a given environment. \cite{b12riand2002assessing}.



- kritische auseinandersetzung, für welche fälle sind welche metriken gut und wieso
- Wann setz ich welche Metriken ein wann im SW Prozess!! und was kann da eintreten passieren
- Auswirkungen von fehlern

Many approaches try to support the software development process with their metrics. All of them work in their own way and in certain situations. 
In order to provide guidance on how to proceed in the software process, some metrics will be explained.

\cite{b7radjenovic2013software}. Context: Software metrics may be used in fault prediction models to improve software quality by predicing fault location. wo wie viele genutzt werden udn so
\section{Summary of Metrics}\label{analysis}

Some fundamental metrics are based simply on counting the number of interactions or the lines of code of a class as shown in Section \ref{related}. Next, we take a closer look at the most famous object-oriented metrics, the "CK-metrics". It is a set of metrics proposed by Chidamber and Kemerer in 1991 specifically for object-oriented software \cite{b15chidamber1991towards}. Build on the basic metrics table of the Chapter \ref{related}, divided in class metrics, method metrics and attribute metrics, more detailed aspects are now examined. 

Many metrics are based on comparable ideas and provide redundant information. By using a subset of metrics, prediction models can be built to identify the classes that are in fault. In the table \ref{tab:metrics}, on can see the CK-metrics summarized, which are named in many other papers too.

\begin{table}
	\caption{The CK Metrics.}~\label{tab:metrics}
	
	\setlength\tabcolsep{3pt}
	\renewcommand{\arraystretch}{1.4}% for the vertical padding
	\begin{tabularx}{\columnwidth}{ | c | p{5.6cm} || c | }
		\hline
		Abbrevation & Definition & Sources \\ \hline\hline
		\textbf{CBO} & Coupling between objects for a class is a count of the number of non-inheritance related couples with other classes & \cite{b15chidamber1991towards, b1aggarwal2007investigating, b3al2012fault} \\ \hline
		\textbf{LCOM} & Lack of Cohesion in Methods counts number of null pairs of methods that do not have common attributes & \cite{b15chidamber1991towards, b1aggarwal2007investigating} \\ \hline
		\textbf{NOC} & The number of children is the number of subclasses of a child class in a hierarchy & \cite{b15chidamber1991towards, b1aggarwal2007investigating} \\ \hline
		\textbf{DOI} & The depth of a class within the inheritance hierarchy is the maximum number of steps from the class node to the root of the tree and is measured by the number of ancestor classes & \cite{b15chidamber1991towards, b1aggarwal2007investigating} \\ \hline
		\textbf{WMC} & The weighted methods per class is a count of sum of complexities of all methods in a class & \cite{b15chidamber1991towards, b1aggarwal2007investigating} \\ \hline
		\textbf{RFC} & The response set of a class is defined as set of methods that can be potentially executed in response to a message received by an object of that class & \cite{b15chidamber1991towards, b1aggarwal2007investigating} \\ \hline
	\end{tabularx}
\end{table}

Next, some of the CK metrics will be presented in more detail and with some examples.

\subsubsection{\textbf{CBO}}: The coupling between objects for a class is a count of the number of non-inheritance related couples with other classes. Two things are coupled if and only at least one of them atcs upon the other. Any evidence of a method of one object using methods or instance variables of another object constitutes coupling. Given is an example for CBO: 

\begin{small}
\begin{verbatim}
import java.util.Calendar;
	
public class AdultIssuePolicy implements
	                              IssuePolicy {
   public Calendar compute(BiblioType type, 
   								Calendar from) {
      Calendar res = (Calendar)from.clone();
        res.add(Calendar.DATE, 14);
        return res;
   }
}
\end{verbatim}
\end{small}

Note, that coupling is not associative, i.e., if $A$ is coupled to $B$ and $B$ is coupled to $C$, this does not imply that $C$ is coupled to $A$.

\subsubsection{\textbf{LCOM}}: The Lack of Cohesion in Methods counts number of null pairs of methods that do not have common attributes. Consider a Class $A$ with methods $m_1$, $m_2$,...$m_n$. Let $\{I_i\} =$ set of instance variables used by the method $m_i$. There are $n$ such sets $I_1$,...,$I_n$. The LCOM is therefore the number of disjoint sets formed by the intersections of the $n$ sets. Formally let´s say a Class $C$ with 
\begin{itemize}
	\item $k$ fields $f_1,f_2,...,f_k$ and
	\item $n$ public methods $m_1,m_2,m_3,...,m_n$.
\end{itemize} 
So $I_i= \{f_l:f_l$ is used by $m_i\}$ and $N=\frac{n(n-1)}{2}$ is the number of different possible pairs of methods. It follows that

\begin{displaymath}
	P=|\{(m_i,m_j):i<j \ and \ I_i \cap I_j = \emptyset\}
\end{displaymath}  
\begin{displaymath}
	Q=|\{(m_i,m_j):i<j\ and \ I_i \cap I_j \neq \emptyset\}
\end{displaymath} 

and $N=P+Q$. The LCOM is $P$. Given is an example for LCOM:

\begin{small}
\begin{verbatim}
public class A {
  private int f1;
  private int f2;
  private int f3;
  private int f4;
  public void method1() { I1 = { f1, f2 }
     // uses f1
     // uses f2
  }
  public void method2() { I2 = { f2, f3 }
     // uses f2
     // uses f3
  }
  public void method3() { I3 = { f3, f4 }
     // uses f3
     // uses f4
  }
}
\end{verbatim}
\end{small}

When looking at the $(m_i,m_j)$ pairs, the $I_i \cap I_j$ of $(method1,method2)$ is $_f2$, $(method1,method3)$ is $\emptyset$ and $(method2,method3)$ is $_f3$. Therefore LCOM is $1$.

\subsubsection{\textbf{NOC}} The number of immediate subclasses subordinated to a class in the class hierarchy, which is pretty much self explanatory.

\subsubsection{\textbf{DOI}} The Depth of inheritance of the class is the depth of inheritance tree metric for the class, it is the length of the longest path to the root. The deeper a class is in the hierarchy, the greater the number of methods it is likely to inherit, making it more
complex, but it is useful to have a measure of how deep a particular class is in hierarchy so that the class can be designed with reuse of inherited methods.

\subsubsection{\textbf{WMC}} The Weighted Methods Per Class is a count of sum of complexities of all methods in a class. Consider a Class $C$ with methods $m_1,m_2,...,m_n$. Let $c_1,..,c_n$ be the static complexity if the methods. Then 
\begin{displaymath}
	WMC = \sum_{i=1}^{n} c_1.
\end{displaymath}

\subsubsection{\textbf{RFC}} the Response For a Class is defined as set of methods that can be potentially executed in response to a message received by an object of that class. The RFC $=|RS|$ where $RS$ is the response set for the class. The response set if an object $\equiv \{$set of all methods that can be invoked in response to a message to the object$\}$. Given is an example for RFC:

\begin{small}
\begin{verbatim}
public class A {
   private B aB;
   public void methA1() {
      return aB.methB1();
   }
   public void methA2(C aC) {
      return aC.methC1();
   }
}
\end{verbatim}
\end{small}

It follows $RS = { methA1, methA2, methB1, methC1}$.

These metrics are the fundamental rules, that developers follow. People who investigate new studies on metrics in objectorineted systems usually have these rules as a cornerstone.
Some issues that comes with this metrics, are what happen with classes with no fields, static member or self calls. Not all questions, provide the metrics answer, they just provide a framework.

Next step, is the look at a study that deals with class cohesion metrics. After that, a bayesian network is explained, which is a bit more difficult.


\subsection{Method-Method Interaction-Based Cohesion Metrics for Object-Oriented Classes}\label{mmi}

Basic units of design in object-oriented programs are classes. Class cohesion refers to the relatedness of class members, i.e., their attributes and methods. Multiple metrics for class cohesion have been proposed in the literature. These object-oriented metrics are based on information available during the high-level or low-level design phases.
A formula that accurately measures the degree of interaction between each pair of methods is proposed and used as the basis for introducing a low-level design class cohesion (LSCC) metric \cite{b8al2012precise}. Low-level design (LLD) cohesion metrics use more finely resolved information than that used by High-level design (HLD) cohesion metrics. HLD cohesion metrics identify potential cohesion issues early in the HLD phase. 
In figure \ref{fig1}, rectangles represent methods, circles indicate attributes, and links illustrate the use of attributes by methods of a class. Metrics based on counting the number of links, i.e., the use of attributes by a method, can indicate whether a class is strongly or weakly cohesive. This finely granulated information is important to help software developers refactoring their code and detecting which methods to possibly remove, i.e., the methods that exhibit even no links with other methods. When a method-method interaction (MMI) metric is applied to measure the cohesion for the class shown in figure \ref{fig1}, the connectivity between each pair of methods is calculated, and it is clearly seen that method $m_3$ is weakly interconnected to other methods in this class.

\begin{figure}[htbp]
	\centerline{\includegraphics[width=0.2\textwidth]{pictures/am.png}}
	\caption{Sample representative graph for a hypothetical class \cite{b3al2012fault}.}
	\label{fig1}
\end{figure}

As shown in figure \ref{fig2}, there are four other classes with a different method-method interconnection. The class cohesion (CC) of Bonja and Kidanmariam is the ratio of the summation of the similarities between all pairs of methods to the total number of pairs of methods \cite{bonja2006metrics}. The similarity between methods $i$ and $j$ is defined as
\begin{displaymath}
	sim(i,j)=\frac{|I_i \cap I_j|}{|I_i \cup I_j|} ,  
\end{displaymath}
where $I_i$ and $I_j$ are the sets of attributes linked by methods $i$ and $j$, respectivly \cite{b3al2012fault}. In contrast with the class cohesion metric of Pena (SCOM), the calculation is defined as follows
\begin{displaymath}
	sim(i,j)=\frac{|I_i \cap I_j|}{min(|I_i|, |I_j|)} \cdot \frac{|I_i \cup I_j|}{n},  
\end{displaymath}
where $n$ is the number of attributes \cite{fernandez2006sensitive}.
Both CC and SCOM neither consider transitive MMI nor account for inheritance or different method types, and they have not been empirically validated against external quality attributes such as fault occurrences. Of course, there are many other metrics of this kind but the results for the metrics that consider the degree of interaction between each pair of methods are very close to each other \cite{b8al2012precise}.

\begin{figure}[htbp]
	\centerline{\includegraphics[width=0.4\textwidth]{pictures/am2.png}}
	\caption{Classes with different method-method connectivity patterns \cite{b3al2012fault}.}
	\label{fig2}
\end{figure}

The results suggest that class quality, as measured in terms of fault occurrences, can be more accurately explained by cohesion metrics that account for the degree of interaction between each pair of methods. The fault prediction of interconnection-based object-oriented class cohesion metrics should help the developer to support refactoring during the LLD phase, therefore in an early state of development \cite{b8al2012precise}.


\subsection{A Bayesian Network}

Some metrics are based on the concept of the possibility that the methods and attributes of a class of the code can be presented as a graph. Here the interaction can be graphically represented by an undirected edge that links a circular node (attribute) to a rectangular node (method).

The next method to take a closer look at the fault-proneness is a concise representation of a joint probability distribution on a set of statistical variables. Bayesian methods can be used for assessing software fault content and fault proneness. A bayesian network (BN) is encoded as an acyclic graph of nodes and directed edges \cite{b9pai2007empirical}. Assuming that the relationship can be modeled with a general linear model, the structural and numerical specification for the BN is derived. The model can be thought of as a generalization of existing techniques for assessing software quality. The model consists roughly of two parts, first the method produces a probability distribution of the estimated fault content per class in the system and second the conditional probability that a class contains a fault. 
The structure of the model is a BN model whose underlying representation is the generalized linear model. The definition probabilistic network (acyclic graph $G=(V,E)$; A set $S$, of (prior) conditional probability distributions).
Consider a finite set of random variables $X=\{X_1,X_2,....,X_n\}$. It can be defined that a probabilistic network $N=(G,X) over X$ consists of
 \begin{enumerate}
 	\item[-] a directed acyclic graph $G=(V,E)$, $V$ is the set of nodes in the graph and there is a one-one correspondence between $V$ and $X$. $E \subseteq V \times V$ the set of directed edges, representing conditional independence assumptions, i.e., for each $X_i \in X$, $i(X_i,N D_{x_i}| Pa_{x_i})$ and $N D_{x_i} = X \backslash ({X_i} \cup Des_{x_i})$ 
 	\item[-] a set of (prior) conditional probability distributions, that specifies $p(X_i)| p(Pa_{x_i})$ for each $X_i \in X$, where $Pa_{x_i}$ represents the set of immediate parents of $X$.
 \end{enumerate} 

Once a network is specified over a set of random variables, their marginal and joint probabilities can be computed.
Given a BN structure, the joint probability distribution over $X$ is encoded as

\begin{displaymath}
	p(X)= \prod_{i=1}^{n}p(X_i|P_{ax_i})
\end{displaymath}

And given this joint probability, the marginal probability of a random variable $X_i$ is computed as 

\begin{displaymath}
	p(X_i)= \sum_{X_i,j\neq i}^{n}p(X)
\end{displaymath}

The CK metrics mentioned in the previous section can be found here, too. The model parameters that are used are listed in the following \cite{b9pai2007empirical}: 

\begin{itemize}
	\item[1.] Weighted methods per class (WMC)
	\item[2.] Depth of inheritance tree (DOI)
	\item[3.] Response for class (RFC)
	\item[4.] Number of children (NOC)
	\item[5.] Coupling between object classes (CBO)
	\item[6.] Lack of cohesion in methods (LCOM)
	\item[7.] Source lines of code (SLOC): This is measured as the total lines of source code in the class and serves as a measure of class size.
\end{itemize}

As can be seen the parameters 1.-6. are the CK metrics \ref{tab:metrics}. The 7. point is already presented in table \ref{tab:classmetrics}, too.
The dependent variables, which serve as surrogate metrics of software quality here, are
\begin{itemize}
	\item Fault Content (\textbf{FC}): We define fault content as the number of faults per class. The estimation of our model is a (marginal) conditional probability of observing a certain number of faults per class, given the metrics for that class.
	\item Fault proneness (\textbf{FP}): The conditional probability that a class contains a fault, given the metrics for that class.
\end{itemize}

In this approach, it is assumed that a general linear model (GLM) relates the dependent and independent random variables and construct a BN structure that represents the GLM. This assumption is motivated by several factors: The GLM is versatile enough to represent existing linear relationships or, through transformations, a variety of nonlinear relationships. Linear models also provide a relatively simple and parameterized way to capture the dependencies between domain variables. 
 
Consider that a responsive variable $Y$ varies as some function of a set $X=\{X_1,X_",...,X_k\}$ of independent predictor variables. In the GLM, it follows 
\begin{displaymath}
 	E(X)= \mu = g^{-1}(X\beta),
\end{displaymath}
where $Y$ is the set of observations with expected value $E(X)= \mu$. The linear predictor with coefficients $\beta$ is $X\beta$ and $g$ is the link function determined by the distribution of $Y$ as graphically shown in figure \ref{fig1bn}. 

\begin{figure}[htbp]
	\centerline{\includegraphics[width=0.2\textwidth]{pictures/bn1.png}}
	\caption{BN representation of a general linear model.}
	\label{fig1bn}
\end{figure}

Figure \ref{fig2bn} shows how the individual components of the model parameters influence the FC and FP. Here the connection between $X$ and $Y$ can be clearly seen.

\begin{figure}[htbp]
	\centerline{\includegraphics[width=0.4\textwidth]{pictures/bn2.png}}
	\caption{BN model for fault content and fault-proneness analysis.}
	\label{fig2bn}
\end{figure}

In table \ref{fig3bn}, the used data is given. After the set up model has gone through the data, one can see, that the SLOC is about $43K$ and the faulty percentage is nearly $40\%$, which is enormous. 
The results also show that this model produces these estimations at a statistically significant level. The results of performing multiple regression, the metrics WMC, CBO, RFC, and SLOC are very significant for assessing both fault content and fault proneness. In general this specific set of predictors is very significant for assessing FC and FP in large software systems.

\begin{figure}[htbp]
	\centerline{\includegraphics[width=0.45\textwidth]{pictures/tableBN.png}}
	\caption{Data Description.}
	\label{fig3bn}
\end{figure}
\section{Discussion}\label{discussion}

As Chapter \ref{analysis} makes clear, many models and methods are based on the fundamentally same metrics introduced at the beginning. It also became clear that it quickly becomes mathematical and complex when, for example, the BN is examined in more detail. Accordingly, finding fault-proneness classes is not an easy task and requires effort and time from the stakeholders and the developers.
In comparison, the individual methods to predict faults have concluded depending on the use case.

When a company or software developer decides to use metrics for their object oriented system to avoid errors, there are many starting points. Some metrics start with an accurate understanding of object orientation, which in itself takes care of fault-proneness. However, just because you think you've found the right metric doesn't mean you won't make any more failures. Another problem, as some papers also note, is that the studies have mostly tested their metrics on small projects and so the validity is not scalable to larger systems. Therefore some aspects such as the size of the software project plays a big role. Some metrics are only designed for smaller systems.

What is certain is that incorporating metrics or certain programming rules is important and can easily be verified via simple metrics to not disregard fault-proneness.


\subsection{Classification of relevance}

There are metrics that can or should be used in an early development process and those that are used in later states. Both do not guarantee fault prediction, but it is important to pay attention to fault-proneness at all.
The aspect of fault-proneness must also be considered during planning and developing to avoid highly costs.

The most influenced persons are the developers but also the customers.
For the developer it is important how to build and construct his object-oriented classes, with a metric as orientation to avoid bugs. For the customers it is of utmost importance that the use of the system runs error free, so the relevance is high.

\subsection{Limitations}

A further problem that has emerged is that many studies were initially only tested on small data sets. The results of the studies are therefore only transferable to this type of software and thus the significance is not automatically scalable to larger data sets. For example, when refactoring classes, not only cohesion must be considered, but also other quality attributes, such as coupling. So it's not enough to get into one specific metric, you should include all of them a bit. However, if you don't have a lot of money to spend, corporations don't want to put their money into research on perfect metrics. The research on fault-proneness metrics so far gives a lot of room to develop new methods and approaches.

Sometimes it can also happen that several classes are displayed as faulty and the cause can only be fixed in a completely different class. This class is thus not discoverable by the applied method and the fault must be searched for elsewhere.

Furthermore all errors that are feasible can never be covered. It depends on the way of programming and the communication among the people working on it, i.e. human influences that no metric can cover. In every context, new situations arise and will continue to arise in the future that require new metrics. If project changes occur in any development process states, the fault prediction metrics may also need to be adjusted, which is not done in most developments.

\subsection{Benefits}

The widespread assumption is wrong that it is enough to look at the fault context once. A person working on the system should always keep an eye on the metrics to be observed in case of changes in order to avoid a costly awakening later on.
Even if you can't be 100\% hedged by metrics on fault-proneness, the ability to make fewer mistakes is a good option for many developers. If fault-proneness is included in early software development steps, a lot of money can be saved.
Later changes to the software, for example, if the architecture has to be changed or the database has to be adapted, are enormously expensive and time-consuming.

It may also happen that by considering design metrics, more attention is paid to building the entire system more carefully and cleanly, which additionally improves maintenance and changeability, among other things.
\section{Conclusion}\label{conclusion}

- summarize discussion and results; name important metrics/properties that should stay in mind after reading this

- what can reduce the fault-pron in software development

- outlook

- This paper discussed the concept of fault proneness using the example of object-oriented programming and design metrics, which means that for the most part the results can only be transferred to the object-oriented context.

- maybe future work

- final catch and the link to the superordinate context (research trends in software technology); link to introduction and motivation

In the field of software evolution, metrics can be used for identifying stable or unstable
parts of software systems

In the area of software reengi-
neering and reverse engineering [7], metrics are being used
for assessing the quality and complexity of software systems,
as well as getting a basic understanding and providing clues
about sensitive parts of software systems.

wie in related work aufgezeigt die flexible metrics die können als ausblick hilfreich sien wenn sich der kontexxt leicht ändert

future work wäre größere systeme anschauen und testen auf faulty classes und metriken testen an sich

%\input{sections/altes}

\bibliographystyle{IEEEtran}
\bibliography{masterseminar}

\end{document}
