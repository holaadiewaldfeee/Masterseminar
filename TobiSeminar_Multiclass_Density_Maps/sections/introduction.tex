%% The ``\maketitle'' command must be the first command after the
%% ``\begin{document}'' command. It prepares and prints the title block.

%% the only exception to this rule is the \firstsection command
\firstsection{Introduction}

\maketitle

%% \section{Introduction} 

A scatterplot is a two-dimensional data visualization that uses dots, squares, crosses, or multiple other kinds of glyphs to represent the values present for two different variables - one plotted along the x-axis and the other plotted along the y-axis. This two dimensional space can either be a geographical space or two coherent dimensions of data. The dots used in scatterplots are commonly representing one single occurrence of a value the data in the background represents.

When these maps scale to larger amounts of data or when the data shown gets an additional dimension a dot distribution map, comes to hand. This is a type of thematic map that most often uses dots but also other symbols or glyphs on the two spatial or variable dimensions to show the values of one or more numeric data fields, or class. With this, quantitative data can be shown in the two dimensional space (x, y) using the domains color, size or opacity to increase the amount of dimensions shown in maps.
Each dot on these dot-density map represents some amount of data where the dots can be encoded using the domains mentioned to show differences. By combining multiple distinctive colors, different scaling of glyphs or masking, these maps can show even more data, such as different classes or properties of fields, it thus gets called a multiclass dot-density map and by using data aggregation in combination with masking or mixing, to show the classes, these are called multiclass density map.

In a density map, areas with many dots, bigger glyphs or specific colors indicate high concentrations of values for a tile of data and fewer dots, smaller glyphs or a different colors indicate lower concentrations.

To explore the expediency of multiclass density maps the second chapter will first compare different approaches to the field of visualizing multiclass data in different ways. The challenges, namely the scalability of data and possible disadvantages in the process of creating density maps will be compared. Also the challenge of class perception and separation will be addressed. In chapter three the taxonomy of these challenges will be further explained and different approaches introduced. Thereafter the class buffer model of Jo et al.~\cite{jo2019declarative} will be introduced and explained and how they approached the tasks presented. Chapter five will finish with discussing the results, advantages and disadvantages of the class buffer model.
% \textcolor{red}{\lipsum[1]}
