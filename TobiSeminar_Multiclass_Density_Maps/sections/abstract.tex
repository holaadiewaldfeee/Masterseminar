\abstract{
	Visualizing data with scatterplots fails to scale as the complexity and amount of information increases. As a consequence, there exist many design options modifying or expanding the traditional graph design to meet these greater scales.
	Multiclass maps are scatterplots, multidimensional projections, or thematic geographic maps where data points have two quantitative attributes in addition to one or more categorical attributes. The latter is frequently rendered by drawing shapes or using colors. These properties unfortunately do not scale well when the data or labels in a data visualization overlap. In this case visualization pipelines and frameworks have to resort to data aggregation to remain comprehensible. For this purpose Jo et al. introduce the Class Buffer model~\cite{jo2019declarative}, which uses multiple 2D histograms, computed for each class of the data, to render \textit{multiclass density maps}.
	This article describes mechanics of this model, what obstacles are to tackle and how the Class Buffer model can be used to create descriptive visualizations of complex data.
}
