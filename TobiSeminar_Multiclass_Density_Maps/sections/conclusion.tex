\section{Conclusion}

Multiclass density maps are an expressive way of displaying data with two quantitative and additional qualitative attributes. They combine the readability of scatterplots with the information entropy of density maps. So far, little has been done regarding the design space and scalability of multiclass maps. This is where the Class Buffer model can be of use.

In the future the Class Buffer model can be extended with more multiclass density map idioms. These can particularly focus on user interaction, enabling panning and zooming the maps or exchanging the colors used in the map. The performance of the implementation can be improved by utilizing WebGL for faster calculations or to support high resolution displays. 

Jo et al.~\cite{jo2019declarative} started from examples of cartography to abstract multiclass density map idioms into a unified Class Buffer model. With this model the design space is widened because of the separation of computation between front-end and back-end. Changing visualization requirements do not have to be binned and aggregated to be shown on a screen. And by using a declarative visualization grammar users can explore the design space of multiclass density maps, finding the best design for achieving their goal.

% \textcolor{red}{\lipsum[2]}
